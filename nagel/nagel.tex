\documentclass{article}

\usepackage[letterpaper]{geometry}
\usepackage{enumerate,verbatim,multirow,color}
\usepackage{fancyhdr}
\usepackage{amssymb}
\usepackage{multicol}
\usepackage{setspace}


%Margin settings
\geometry{hmargin={1in,1in},vmargin={1in,1in}}
\setlength{\parindent}{1cm}
\setlength{\parskip}{0mm}
\setlength{\tabcolsep}{10pt}
\setlength{\arraycolsep}{10pt}
\renewcommand{\baselinestretch}{1.75}

%Header settings
%\setlength{\headheight}{15pt}
\pagestyle{fancy}
\fancyhead{}
\fancyhead[L]{Phil 100, f15}
\fancyhead[C]{Example reading response}
\fancyhead[R]{Hoversten}

%Hyperref
\usepackage[colorlinks=true,linkcolor=blue]{hyperref}

\begin{document}

{\setlength{\parindent}{0cm}
Erik Hoversten

\vspace{-3mm}

Mind reading response

\vspace{-3mm}

September 1, 2015

\vspace{-3mm}
}

\begin{center}
\subsection*{Thomas Nagel, \textit{What is it like to be a bat?}}
\end{center}

\subsubsection*{Exegesis}

In \emph{What is it like to be a bat}, Thomas Nagel explores the
mind-body problem, which can be understood as the challenge of
explaining how mental things like ideas are connected to physical things
like the human brain. One possible such connection is that of
\emph{reduction}, which the Norton editors describe in a footnote as
``roughly put, to say that A can be reduced to B is to say that A is
nothing over and above B'' (p. 403, fn. 1). The way I understand this
idea in application to the mind-body problem is that the mind can be
reduced to the physical if we can \emph{explain} everything about the
mind in purely physical terms. An example of successful reduction is our
ability to understand all there is about rainbows in terms of light
waves refracting through water droplets in the air.

Nagel thinks that \emph{consciousness} poses a problem for the attempt
to reduce the mental to the physical. He thinks that if we try to
explain consciousness purely in terms of brain functioning, we will
leave out something crucial. Thus, the explanation won't be complete,
and the mental must be something over and above physical stuff.

Nagel's motivation for this suggestion is his story about the bat. He
asks us to try to imagine what it is like for the bat. Crucially, this
is not to imagine ourselves if we had wings and echolocation. This is
just transferring our own conscious experience into a bat-like context.
Instead, we are to attempt to imagine what it is like \emph{for the bat
itself}. Nagel maintains, and he thinks we will agree, that this is
beyond our imaginative capabilities.

The problem with trying to undertake this task is that it requires us to
shift \emph{perspective} from our own human context to that of a bat.
Nagel suggests that perspective is an essential element of
consciousness, so to shift perspective is to alter consciousness. We
can't imagine what it is like for a bat because that would require us to
give up our own consciousness and take on the bat's. But we can't give
up our consciousness without giving up the very project of explaining
things to ourselves.

So, Nagel's diagnosis of the mind-body problem is that the mind's
perspective is one of its essential elements. And this explains why
reduction is going to fail. In general, science and reduction are about
widening our perspective, eliminating the specifics of our own situation
in order to reach an objective understanding of the thing in question.
But the bat example shows that first-person perspective is essential to
fully understanding consciousness. Thus, the very nature of reduction
(eliminating first-person perspective) makes it bound to fail when it
comes to explaining all there is to consciousness.

Nagel is careful to point out that his idea isn't that physicalism is
false, only that perspective can sometimes be crucial to the thing we
are investigating. So, if physicalism is to be vindicated, we need a
method of characterizing perspective in physical terms. And this, we
just don't yet have.

\subsubsection*{Critique}

I find Nagel's argument to be very compelling. I see it as having 2
major premises:

\begin{enumerate}
\itemsep1pt\parskip0pt\parsep0pt
\item
  Reduction is a matter of moving from a more subjective perspective on
  something to a more objective perspective. In so moving, one leaves
  the subjective perspective behind.
\item
  The first person viewpoint is a subjective perspective that is an
  essential element of conscious experience.
\end{enumerate}

The conclusion is that reduction cannot hope to explain all of conscious
experience. As for the first premise, I don't know if there is another
way of understanding reduction than by appealing to perspective
shifting. I'm going to leave that for someone else to discuss. In this
critique, I'm going to focus on the second premise, and specifically the
idea of perspective being an essential \emph{element} of consciousness.

My main question is what \emph{perspective} amounts to. It doesn't seem
that we can think of perspective as a \emph{part} of the mind in the way
that peanuts are a part of pad thai. Perspective isn't really a thing
that you add to something else. Instead, perspective is something that
is external to the thing that has the perspective. It is something like
a relation between an observer and a thing that they observe. If this is
right, then it seems that acknowledging the significance of perspective
requires us to acknowlege the observer that has the perspective. And
that observer is presumably a mental entity. But then, the challenge for
physicalism isn't about capturing perspective, but making sense of the
mental thing that has a perspective.

On the other hand, Nagel seems to be saying that consciousness
\emph{just is} having a perspective on the world. In this case, maybe we
can think of perspective as an added thing in the world. You take some
physical stuff, like neural firings in the brain, and add a perspective
to it to get something mental.

But now I wonder if equating the essential bit of mentality with a
perspective on reality itself reduces the mind too far. When I think of
my thoughts and emotions, I tend to think of them as independent things
populating the world and interacting with it. I can imagine them
extending out beyond me and having an impactful life in the world. Maybe
my thoughts originate in my head, but I don't think of them as confined
to that space. The idea of being able to read someone's mind seems to me
to require that thoughts can escape one head and enter another.

This idea seems close to what Descartes was talking about in his
\emph{Meditations}. He thinks of mental stuff and physical things as two
different kinds of \emph{substance} that somehow interact with
eachother. On Nagel's account, though, we can't think of the mental as
independent and able to escape one's head. This is because all that
makes something mental is the first-person perspective, which is limited
to the head of the individual thinker. If the mental thing tried to
escape, it would be severed from the perspective and cease to be mental
at all. It would be lost entirely, or at least altered to the point that
it couldn't be recognized as the idea it started out as.

I guess my point is that if we think of thoughts as shareable between
individuals, then they must have an ability to exist independently of
any one perspective. On Nagel's account as I understand it, this
independent existence is not possible, and this seems to me to be a
point of concern for his perspective based view of consciousness.

\end{document}
