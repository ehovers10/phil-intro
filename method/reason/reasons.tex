\documentclass[10pt]{article}

\usepackage[letterpaper]{geometry}
\usepackage{enumerate,verbatim}
\usepackage{fancyhdr}
\usepackage{linguex}
\usepackage[colorlinks=true,linkcolor=blue]{hyperref}

%Packages needed for trees
\usepackage{amsfonts,amsmath,amssymb}
\usepackage[varg]{txfonts}
\usepackage{qtree}

%% Margin Setting
\geometry{hmargin={.5in,.5in},vmargin={1in,1in}}
\setlength{\parindent}{0.0in}
\setlength{\parskip}{3mm}
\setlength{\tabcolsep}{10pt}
\setlength{\arraycolsep}{10pt}

%% Header
\pagestyle{fancy}
\fancyhead{}
\fancyhead[L]{Phil 101, f14}
\fancyhead[C]{Reasons}
\fancyhead[R]{Hoversten}

\begin{document}

\begin{comment}
\section{Arguments}
An argument is a set of \textbf{propositions} (sentences or statements), one of which is designated the \textbf{conclusion}, and the rest of which are called the \textbf{premises}.\footnote{For clarity, it is often helpful to present arguments with the propositions written in list format and each of the premises numbered.} To endorse an argument is to maintain that the premises provide some sort of \textbf{support} for the conclusion.  That is, if the argument is any good, then a \textbf{rational} person who believes each of the premises would also be inclined to believe the conclusion.

Engaging in the practice of providing arguments for our beliefs thus appeals to the rationality of the people with whom we are engaged in debate.  Of course, nothing stops a person from accepting that the argument is a good one while still refusing to believe the conclusion.  But in so far as the person has an interest in having reasons for the beliefs they hold, recognition of good arguments should result in appropriate changes in their beliefs.

Arguments provide full fledged reasons to take on particular beliefs or to act in particular ways.  But frequently when we start an inquiry, we may not have well worked out arguments.  Instead, we look for what we might call \textbf{considerations}.  Considerations are ideas that seem to speak in favor of one convclusion over another.  Once we get a bunch of considerations on the table, we can begin sorting through them and see whether the balance of considerations does support one conclusion over the other and how strongly.
\end{comment}

\section{Kinds of reasons}
The word \textit{reason} can be used in many different ways.  To help get a grip on what it means to engage in the practice of giving and receiving reasons for our moral beliefs, let's examine some of the different kinds of reasons we might profer for a position.


\subsection{Epistemic v. Pragmatic}
Imagine that you are being chased through the woods by a bear.  All of a sudden, you break into a clearing and approach the edge of a chasm.  The chasm looks to be about 20ft wide. It occurs to you that if you try to change course, you will most certainly be devoured by the bear.  But you also recall that the best long jumpers in the world rarely jump more than 25ft, and you are no world class athlete. You are in quite a predicament.

In this scenario, both of the following claims seem to be true.
\ex. You have reason to believe that you cannot make the jump.

 \begin{itemize}
  \item All of your evidence suggests that the distance is too far for you to cross in a single leap.
 \end{itemize}
 
\ex. You have reason to believe that you can make the jump.

 \begin{itemize}
  \item If you don't jump, you're gonna get eaten.  And maybe believing you can make it will give you a slightly better chance of doing so. So, you might as well believe it.
 \end{itemize}

On their face, claims \LLast and \Last are in direct contradition. The fact that they both seem true suggests that there is an ambiguity in our use of the word ``reason''.

Let's use the term \textbf{epistemic reason} to refer to the kind of reason mentioned in \LLast.  Epistemic reasons appeal to our cognitive faculties and purport to support the \textbf{truth} of some claim.  And let's call the kind of reason mentioned in \Last \textbf{pragmatic reason}.  Pragmatic reasons appeal to our connative faculties (our senses of emotion and well-being) and purport to show that we \textbf{benefit} in some way from believing the claim.

\subsection{Explanatory v. Justificatory}
One role that reasons play is that they serve as answers to \textit{Why?} questions.  But \textit{Why?} questions can be ambiguous.  Consider:

\ex. Why did Frank take the marble rye home?
\a. Because he paid for it.
\b. Because he thought no one was looking.

Imagine that \Last[a] is true. Then it provides a reason for why Frank took the beer, and supposing that having paid for something gives someone the right to do with it what they please, it is a reason that \textbf{justifies} Frank's behavior.

But now suppose that \Last[b] is true (and \Last[a] is false). Then, this also provides a reason for Frank's behavior, but it doesn't seem to justify that behavior.  Instead, one might offer this reason as a way of \textbf{explaining} why Frank did it.


\subsection{First blush v. All things considered}

Holmes is investigating a crime scene. He first comes upon a blond hair. He knows that of his two suspects, only Sally has blond hair, whereas Dirk is a brunette.  This leads Holmes to say:

\ex. I now have reason to believe that Sally is the culprit.\label{hair}

As he investigates further, he finds a solid fingerprint.  His friend at the police station tells him that there is no strong correlation with the fingerprint of Sally's that they have on file.  Since fingerprint evidence is more weighty than hair folicles,\footnote{How much more weighty? It's not clear.  \href{http://en.wikipedia.org/wiki/Finger_Printing\#Validity}{Finger printing} is less of a science than many people realize.} Holmes now says:

\ex. Now the balance of reasons suggests that Sally did not do it.

More searching uncovers a bottle of brown hair dye in the trash and a man's footprint at the scene. Having completed a thorough search, Holme's exclaims:

\ex. Aha! It wasn't Sally at all, but Dirk!

In this example, \ref{hair} seems accurate at first blush.  But two things happen as Holmes looks into the case in more detail.  First he gets \textbf{rebutting evidence} against \ref{hair}.  The fingerprint evidence stands in direct conflict with what the hair seems to show.

Then he gets \textbf{under cutting} evidence against \ref{hair}. When he discovers the hair dye, he realizes that a blond hair doesn't support Sally's guilt at all, since Dirk very well could have been blond, too.

Finally, having put all the pieces together, Holmes forms an \textbf{all things considered} opinion that Dirk is the criminal.

\end{document}
