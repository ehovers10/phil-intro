\documentclass[10pt,letterpaper,xcolor=dvipsnames,handout]{beamer}

%\usepackage[colorlinks=true,linkcolor=blue]{hyperref}
\usepackage{amssymb,mathabx}
\usepackage{linguex}
\usepackage{verbatim,enumerate,multirow}
\usepackage{xcolor}
%\usepackage{floatflt}

\usepackage{pgfpages} % to put several slides on one page
\pgfpagesuselayout{4 on 1}[letterpaper, landscape, border shrink=5mm]

\mode<article>{}

%Template themes
\usetheme{boxes}
%\useoutertheme{miniframes}
\useoutertheme{shadow}

%Templates
\setbeamertemplate{blocks}[rounded][shadow=true]
\setbeamertemplate{navigation symbols}[vertical]
\setbeamertemplate{section in head/foot shaded}[default][20]
\setbeamertemplate{title}

\setbeamertemplate{headline}
{
	\begin{beamercolorbox}[ht=3ex,dp=1ex]{erikcolor1}
		\insertshorttitle
		%\insertsectionnavigationhorizontal{\textwidth}{}{}
		%\usebeamerfont{title in head/foot}
	\end{beamercolorbox}
	\begin{beamercolorbox}[ht=3ex,dp=2ex]{erikcolor2}
	  \insertsectionnavigationhorizontal{\textwidth}{}{}
		%\insertsubsectionnavigationhorizontal{\textwidth}{}{}
		%\insertsubsection
	\end{beamercolorbox}
}
\setbeamertemplate{footline}
{
	\begin{beamercolorbox}[ht=3ex,dp=1ex]{erikcolor1}
		\insertshortinstitute[width=.33\textwidth,center] %$\triangleright$
		\insertshortsubtitle[width=.33\textwidth,center] %$\triangleright$
		\insertshortdate[width=.20\textwidth,center] %$\triangleright$
		\hfill\insertframenumber\,/\,\inserttotalframenumber\;\;\;
	\end{beamercolorbox}
}

%Font themes
\usefonttheme{structuresmallcapsserif}
%\usefonttheme[onlysmall]{structurebold}
\usefonttheme{serif}

%Color themes
%\usecolortheme{beetle}
%\usecolortheme{rose}

%Head and foot lines colors
\setbeamercolor{erikcolor1}{fg=white,bg=blue!70!green}
\setbeamercolor{erikcolor2}{fg=white,bg=blue!60!green!10!white}

%Titles color
\setbeamercolor{frametitle}{fg=black,bg=green!30!blue!30!white}
\setbeamercolor{title}{fg=black,bg=green!30!blue!30!white}

%Block color
\setbeamercolor{block title}{fg=white,bg=blue!70!green}
\setbeamercolor{block body}{fg=black,bg=green!30!blue!30!white}

%Background color
\setbeamercolor{background canvas}{bg=}

%Covered items color
\setbeamercovered{transparent}

\AtBeginSection[]
{
   \begin{frame}<beamer>
       \frametitle{Lecture plan}
       \tableofcontents[currentsection,currentsubsection]
   \end{frame}
}



\title{Recognizing, categorizing, and assessing arguments}
\subtitle{Basic tools}
\author[Hoversten]{Erik Hoversten}
\institute[lrp-f14]{Logic, reason, and persuasion: fall 2014 \\ Rutgers University}
\date[09/10/2014]{September 10, 2014}

\begin{document}

\begin{frame}
\titlepage
\end{frame}

\section{Recognizing arguments}

\begin{frame}
\frametitle{The meanings of statements}

\begin{itemize}
  \item Statements are sentences that are put forward with \textbf{assertoric} force.
  \item Sentences can also be put forward as \textbf{questions}, \textbf{suppositions}, or \textbf{commands}.
  \item A propositions is the \textbf{content} of a sentence; the meaning to which different force can be applied.
\end{itemize}

\begin{block}<2->{Example}
\begin{tabular}{lll}
Sentence & Force & Content \\ \hline
``The door is ajar.'' & statement & \textit{the door is ajar} \\
``Is the door ajar?'' & question & \textit{the door is ajar} \\
``I wonder if the door is ajar.'' & supposition & \textit{the door is ajar} \\
``Shut the door!'' & command & \textit{the door is ajar} \\
\end{tabular}
\end{block}

\uncover<3->{\textbf{Assertoric} force is the linguistic force whereby one makes a claim about the world.}

\end{frame}

\begin{frame}
\frametitle{Arguments in general}

\begin{block}{Arguments are:}
\begin{enumerate}
  \item sets of propositions 
  \item where the premises stand in a relation of support to the conclusion, and
  \item the conclusion is put forward with assertoric force
\end{enumerate}  
\end{block}

\begin{block}<2->{Not just anything is an argument}
\begin{itemize}
  \item It's not always obvious how to reconstruct an argument in our sense from a bit of prose that we may be presented with.
  \item Even if the writing is clear enough that we can identify the propositions each statement makes, not all sets of propositions constitute arguments.
\end{itemize}
\end{block}

\end{frame}

\begin{frame}
\frametitle{Examples of non-arguments}

\begin{block}{No supporting premises}
  \begin{itemize}
    \item Bald claims: New Jersey is better than New York in every way.
    \item Opinion: Anyone who can root for the Yankees has got a view of baseball that I can't understand.
  \end{itemize}
\end{block}

\begin{block}<2->{No relation of support}
\begin{itemize}
   \item Report, exposition, illustration
   \item These provide a list of statements, but don't impose a structure on them.
\end{itemize}
\end{block}

\begin{block}<3->{Lack appropriate force}
  \begin{itemize}
    \item Warnings: Whatever you do, don't go to the Golden Rail on gamedays.
    \item Advice: Before taking on a job during the semester, you should get a sense of your full course load.
  \end{itemize}
\end{block}

\end{frame}

\begin{frame}
\frametitle{Premise and conclusion indicators}

\begin{block}{Common premise indicators}
\begin{itemize}
  \item Since...
  \item Because of...
  \item In light of the fact that...
  \item It's well established that...
\end{itemize}
\end{block}

\begin{block}<2->{Common conclusion indicators}
\begin{itemize}
  \item Therefore... Thus... So...
  \item It follows that...
  \item One can conclude that...
  \item $\therefore$
\end{itemize}
\end{block}

\end{frame}

\begin{frame}
\frametitle{Recognizing deductive arguments}

\begin{block}{Common deduction indicators}
\begin{itemize}
  \item Necessarily...
  \item It follows directly...
  \item It must therefore be that...
\end{itemize}

\end{block}

\begin{block}<2->{Deductive argument types}
\begin{itemize}
  \item Definitions
  \item Categorical syllogism
  \item Hypothetical syllogism
  \item Disjunctive syllogism
\end{itemize}
\end{block}

\end{frame}

\begin{frame}
\frametitle{Recognizing inductive arguments}

\begin{block}{Common induction indicators}
\begin{itemize}
  \item Probably...
  \item There's a good chance that...
  \item Most likely...
\end{itemize}

\end{block}

\begin{block}<2->{Inductive argument types}
\begin{itemize}
  \item Prediction
  \item Analogy
  \item Generalization
  \item Causal inference
\end{itemize}
\end{block}

\end{frame}

\section{Propositions}

\begin{frame}
\frametitle{Atomic propositions}

\begin{itemize}
  \item Atomic propositions are the basic building blocks of arguments.
  \item They are composed of a \textbf{subject} and a \textbf{predicate}.
  \item They are called \textbf{particular} if their subject is a specific individual.
  \item They are called \textbf{general} if their subject is a group of individuals or an unspecific individual.
\end{itemize}

\begin{block}<2->{Examples}
\begin{itemize}
  \item John is American. \uncover<3->{\textcolor{red}{(particular)}}
  \item Mary went to the bank. \uncover<3->{\textcolor{red}{(particular)}}
  \item Dogs have tails. \uncover<3->{\textcolor{red}{(general - group)}}
  \item The first person on Mars will be very brave. \uncover<3->{\textcolor{red}{(general - unspecific)}}
  \item Some swans are black. \uncover<3->{\textcolor{red}{(general - group and unspecific)}}
\end{itemize}
\end{block}

\end{frame}

\begin{frame}
\frametitle{Compound propositions}

\begin{block}{Negations...}
\begin{itemize}
  \item are composed of a negation word (not, no, un-) plus an atomic proposition.
  \item have the opposite truth value of the atomic proposition.
  \item \textit{It is not the case that} the Moon is made of cheese.
\end{itemize}
\end{block}

\begin{block}<2->{Disjunctions...}
\begin{itemize}
  \item are composed of two atomic propositions connected by ``or''.
  \item are true just in case \textit{either} of the atomic propositions are.
  \item John is stuck in traffic \textit{or} he overslept.
\end{itemize}
\end{block}

\begin{block}<3->{Conjunctions...}
\begin{itemize}
  \item are composed of two propositions connected by ``and''.
  \item are true only if \textit{both} of the atomic propositions are.
  \item Susan had the polenta \textit{and} Willie had the eggplant.
\end{itemize}
\end{block}

\end{frame}

\begin{frame}
\frametitle{Compound propositions}

\begin{block}{Conditionals}
\begin{itemize}
  \item Conditionals are composed of two atomic propositions connected by ``if $\ldots$ then $\underline{\;\;\;}$''.
  \item we call the first proposition the \textbf{antecedent}.
  \item we call the second proposition the \textbf{consequent}.
\end{itemize}
\end{block}

\begin{block}<2->{Sufficient condition}
\begin{itemize}
  \item The conditional says that whenever you have the antecedent, that is enough to get the consequent. This means that the \textit{antecedent} is a \textbf{sufficient} condition for the consequent.
\end{itemize}
\end{block}

\begin{block}<3->{Necessary condition}
\begin{itemize}
  \item The conditional says that you can't have the antecedent without also having the consequent. This means that the \textit{consequent} is a \textbf{necessary} condition of the antecedent.
\end{itemize}
\end{block}

\end{frame}

\begin{frame}
  \frametitle{Example arguments with conditionals}
  
  \begin{block}{Modus ponens}
  \begin{enumerate}
    \item If it rained last night, then the sidewalks are wet.
    \item It rained last night. \uncover<2->{\textcolor{red}{(\textit{sufficient condition})}}
    \item $\therefore$, the sidewalks are wet.
  \end{enumerate}
  \end{block}
  
  \begin{block}{Modus tollens}<3->
  \begin{enumerate}
    \item If it rained last night, then the sidewalks are wet.
    \item The sidewalks are \textbf{not} wet. \uncover<4->{\textcolor{red}{(\textit{necessary condition})}}
    \item $\therefore$, it did \textbf{not} rain last night.
  \end{enumerate}
  \end{block}
  
\end{frame}



\section{Evaluating arguments}

\begin{frame}
\frametitle{Deductive arguments}

\begin{block}{Validity (measures the \textit{support} relation)}
\begin{itemize}
  \item Deductive arguments purport to give conclusive reason to believe the conclusion.
  \item If the argument is a good one, the conclusion follows directly from the premises.
  \item We say that a deductive argument is valid just in case \textbf{if the premises are all true, then the conclusion must be true.}
\end{itemize}
\end{block}

\begin{block}<2->{Soundness (measures the \textit{truth} of the premises)}
\begin{itemize}
  \item But not all valid arguments have true premises.
  \item If a valid argument also has true premises, we say that it is \textbf{sound}.
  \item If we have a sound argument, we know for certain that the conclusion is true.
\end{itemize}
\end{block}

\end{frame}

\begin{frame}
  \frametitle{Example problematic deductive arguments}
  
  \begin{block}{Invalid argument}
  \begin{enumerate}
    \item All penguins are birds. \uncover<2->{\textcolor{red}{(True)}}
    \item Some birds fly. \uncover<2->{\textcolor{red}{(True)}}
    \item $\therefore$, penguins fly. \uncover<2->{\textcolor{red}{(False)}}
  \end{enumerate}
  \end{block}
  
  \begin{block}<3->{Valid but unsound argument \uncover<4->{\textcolor{red}{(disjunctive syllogism)}}}
  \begin{enumerate}
    \item Either Eli Manning is a running back or dolphins are fish. \uncover<4->{\textcolor{red}{(False)}}
    \item Dolphins are \textbf{not} fish. \uncover<4->{\textcolor{red}{(True)}}
    \item $\therefore$, Eli Manning is a running back. \uncover<4->{\textcolor{red}{(False)}}
  \end{enumerate}
  \end{block}
\end{frame}

\begin{frame}
\frametitle{Inductive arguments}

\begin{block}{Strength (measures the \textit{support} relation)}
\begin{itemize}
  \item Inductive arguments purport to give reason to believe that the conclusion is probable.
  \item The better the argument is, the more probable the conclusion becomes.
  \item We say that an inductive argument is \textit{strong} just in case \textbf{if the premises are all true, then the conclusion is very likely to be true.}
\end{itemize}
\end{block}

\begin{block}<2->{Cogency (measures the \textit{truth} of the premises)}
\begin{itemize}
  \item But not all strong arguments have true premises.
  \item If a strong argument also has true premises, we say that it is \textbf{cogent}.
  \item If we have a cogent argument, we know that the conclusion very probably true.
\end{itemize}
\end{block}

\end{frame}

\begin{frame}
  \frametitle{Example problematic inductive arguments}
  
  \begin{block}{Weak argument \uncover<4->{\textcolor{red}{(does not provide a representative sample)}}}
  \begin{enumerate}
    \item I know a lawyer who is a total liar. \uncover<2->{\textcolor{red}{(True)}}
    \item Bill is also a lawyer. \uncover<2->{\textcolor{red}{(True)}}
    \item $\therefore$, Bill is probably a liar. \uncover<2->{\textcolor{red}{(False)}}
  \end{enumerate}
  \end{block}
  
  \begin{block}<3->{Strong but not cogent argument \uncover<4->{\textcolor{red}{(the study was inaccurately run)}}}
  \begin{enumerate}
    \item 89\% of male college basketball players are over 6 feet tall. \uncover<4->{\textcolor{red}{(False)}}
    \item Bill is a college basketball player. \uncover<4->{\textcolor{red}{(True)}}
    \item $\therefore$, Bill is probably over 6 feet tall. \uncover<4->{\textcolor{red}{(False)}}
  \end{enumerate}
  \end{block}
\end{frame}

\end{document}
