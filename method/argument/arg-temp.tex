\section{Arguments}

\begin{frame}
\frametitle{Arguments as reasons}

\begin{enumerate}
  \item In this class, we're going to try to put all our \textbf{reasons} that we give for various beliefs into the form of an \textbf{argument}.
  \item<2-> The purpose of this practice is to make it clear what exactly we are \textbf{offering} as our reason, and also to make it easier to \textbf{assess} whether our reason is a good one.
  \item<3->\href{https://www.youtube.com/watch?v=kQFKtI6gn9Y}{So, what is an argument?}
\end{enumerate}
  
\end{frame}

\subsection{Basic structure of arguments}

\begin{frame}
\frametitle{Argument form}

\begin{block}{A philosophical argument is...}
  \begin{itemize}
    \item<2-> A set of propositions
      \begin{itemize}
        \item One of these is labeled the \textbf{conclusion}.
        \item The rest are called the \textbf{premises}.
      \end{itemize}
    \item<3-> A relation among the propositions
      \begin{itemize}
        \item The premises are said to \textbf{support} the conclusion.
        \item We can also say that the conclusion is believed as an \textbf{inference} from the premises.
      \end{itemize}
  \end{itemize}
\end{block}
  
\uncover<4->{In order to provide a complete reason in favor of your conclusion, you need to provide both the supporting premises \textbf{and} the nature of the support those premises provide for the conclusion.  When you've done that, you've given your \textbf{reasoning} in favor of the conclusion.} 
 
\end{frame}

\begin{frame}
\frametitle{Example argument}

\begin{block}{Example}
  \begin{itemize}
    \item Premise 1: John can only be in one of three places: his office, the common room, or his classroom.
    \item Premise 2: He's not in his office.
    \item Premise 3: He's not in the common room.
    \item Conclusion: So, John is in his classroom.
  \end{itemize}
\end{block}

\uncover<2->{The inference in this argument is a \textbf{process of elimination}.}

\end{frame}

\subsection{Types of arguments}

\begin{frame}
\frametitle{Types of argument}

We can distinguish different kinds of argument based on the nature of the support the premises are said to provide for the conclusion.

\begin{block}<2->{Deductive}
  In a deductive argument, the premises provide \textbf{conclusive} support for the conclusion.  If the premises are actually true, then \textbf{there's no way} the conclusion can be false.
\end{block}

\begin{block}<3->{Inductive}
  In an inductive argument, the premises make it \textbf{more likely} that the conclusion is true.  But they don't guarantee that the conclusion is true.
\end{block}

\begin{block}<4->{Abductive}
  In an abductive argument, the premises provide a \textbf{good explanation} of why the conclusion is true. But even a good explanation can sometimes be incorrect.
\end{block}

\end{frame}

\begin{frame}
\frametitle{Examples}
\small

\begin{block}<2->{Deductive}

\end{block}

\end{frame}

\begin{frame}
\frametitle{Uses for the argument types}

\begin{block}{Deductive}
Deductive arguments are useful in mathematics, and perhaps in legal settings where the standards of argumentation are incredibly high.
\end{block}

\begin{block}{Inductive}
Inductive arguments are useful in the sciences, and in applications like weather forcasting, where we want to know the chances of an outcome even if we can't be certain that it will happen.
\end{block}

\begin{block}{Abductive}
Abductive arguments are useful in the sciences, and in group brainstorming sessions, where we may not even know what all the alternatives are, and we want to get the options out on the table.
\end{block}

\end{frame}

\begin{frame}
\frametitle{The good, the bad, and the ugly}
\small

\begin{block}{Today, we looked at a \textbf{taxonomy} (categorization) of reasons and arguments.}
\begin{itemize} 
  \item But we shouldn't think of one kind of argument as \textit{better} than another.
  \item Different argumentation techniques are appropriate for different situations.
  \item What is important is that we try to be clear on what sort of reason we are faced with, so that we can better understand how to revise our beliefs in response to it.
\end{itemize}
\end{block}

\begin{block}<2->{But not all arguments are created equal}
\begin{itemize}
  \item Over the course of the semester, we will examine ways of assessing arguments to determine whether we should be swayed by them.
  \item But we can only do this if we take the time to lay out the argument faithfully and understand what kind of reason our fellow inquirer has presented us with.
\end{itemize}
\end{block}

\end{frame}

\begin{frame}
\frametitle{Next meeting}

Next meeting we're going to look more closely at argument structure. We'll examine:
\begin{itemize}
  \item Types of propositions
  \item Properties of arguments
  \item Basic assessment criteria for arguments
\end{itemize}

\begin{block}<2->{Relevant reading}
  Hurley: \S\S 1.3-1.4
\end{block}

\end{frame}

